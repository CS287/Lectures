
\documentclass[11pt]{article}
\usepackage{common}
\usepackage{tikz}
\title{Syllabus for \\ CS 287: Statistical Natural Language Processing}
\date{}

\begin{document}

\maketitle{}

\vspace{-1.75cm}
\section{Overview}

CS 287r is a graduate introduction to statistical natural language
processing, i.e. the analysis and transformation of written language
by computational methods.  Natural language processing (NLP) aims
to create general representations of text that can aid prediction,
extraction, and semantic reasoning over language. Recent consumer
developments in NLP include automatic language translation, hand-held
personal digital assistants, and the extraction of structured
knowledge bases from the web.

Modern statistical NLP is highly intertwined with the field of machine
learning (ML). NLP provides a rich collection of challenging and
large-scale applications , whereas ML provides a formal vocabulary and
set of techniques for statistical modeling. Over the last two decades
developments in both areas have led to major measurable progress
towards computational understanding of language. 

The interplay between ML and NLP will be the central focus of this
class. Much of the course will concentrate on developing applied
mastery of the central models and algorithms of machine learning, such
as multinomial models, multi-class logistic regression, hidden Markov
models (HMM), and, particularly this year, a focus on deep
learning/neural network models such as log-bilinear models (LBL),
convolutional networks (CNN), and recurrent neural networks (RNN).  We
will explore the various benefits and downsides of these models with
both mathematical analysis and hands-on experimental work.

Despite the general effectiveness of these methods, throughout the
class we will often see that the naive application of ML to NLP often
leads to significant computational and modeling challenges. For some
of these challenges we will see clean eloquent solutions; for others
we will see only open questions and opportunities for new
developments. 

In particular, three core properties of text will guide the
overarching questions of our study: (1) Text is a discrete
system. What higher-level representations of text can be used by
statistical models? How can these representations be improved by
data-driven approaches? (2) Text is a structured system. How can we go
beyond system of regression and classification to predict complete
structured outputs such as translations or syntax? (3) Text is a
symbolic system. How do we account for phenomenon like reference and
dependency relationships within a sentence or document? When is it
necessary to model these directly? By the end of the class students
will have seen several different approaches for dealing with these
challenges, and will be able to develop their own methods for
overcoming them.


% \end{itemize}
 

% In addition to introducing algorithms that every computer scientist
% should know, the course will provide a good foundation for topics
% covered in advanced AI courses (CS28x). CS 182 complements CS 181,
% which emphasizes machine learning. Students who take both CS 182 and
% CS 181, will have the background for understanding current artificial
% intelligence research and experience implementing algorithms and
% developing domain models in all key areas of the field.


% The course has three core sections: search, representation, and
% uncertainty.  In each section, it will provide a thorough
% understanding of major approaches, representational techniques and
% core algorithms. In particular we will focus on the trade-offs between
% the model structure of different frameworks and the algorithmic
% constraints that this structure implies. Central topics in
% \textbf{search} will include classical search algorithms, heuristics
% and relaxation, and adversarial game-playing.  The
% \textbf{representation} section will cover constraint-satisfaction,
% logical formalisms for representing knowledge, efficient algorithms
% for logical inference, and an introduction to planning. The section on
% \textbf{uncertainty} will introduce probabilistic reasoning, the
% formalism of Markov decision processes, and conclude with an overview of 
% Bayesian networks for modeling uncertainty.

% Within each area, the course will also present practical AI algorithms
% being used behind the scenes every day and explore cutting edge
% research and philosophical foundations.  The class will include
% lectures connecting the models we explore with application in natural
% language processing, vision, machine learning, and robotics.


% Finally, in spite of its practical usefulness this course is also
% quite fun. AI also has a long history of research into topics like
% puzzle-solving, game-playing, and conversational chat-bots. In this
% spirit, assignments will include programming an efficient Sudoku
% solver, a clever maze-solving robot, and a ghost-avoiding agent for
% Pac-Man. There will also be much XKCD.



\paragraph{Objectives}

Students completing this course will have a the background to read,
implement, and extend state-of-the-art research in natural language
processing. They should be able to:

% Students completing this course will have an in-depth understanding of
% three core areas of AI and the connections among them, and with such
% other key AI areas as machine learning, robotics, natural language
% processing and multi-agent systems. They should be able to:

\begin{itemize}
\item develop formal models to express natural language phenomenon
\item utilize mathematical language to describe algorithms for
  language processing
\item implement and debug large NLP systems in a clean and structured
  manner
\item design and analyze the computational performance of the algorithms presented in the class  
\item describe the results of statistical systems in a logical and
  empirical way, both in writing and orally
\item critically read papers from NLP and ML conferences
\end{itemize} 

Finally, the main assignment for the class will be a final project due
in May.  We expect the final project to be a significant research
project aspiring to conference publication level. We note that there 
are often conference deadlines for EMNLP and NIPS (ML) in early June. 

\section{Preliminaries} 

\paragraph{Prerequisites}

CS 181, CS 281, or Stat 110, as well as significant programming
experience.  No previous exposure to NLP is assumed. Talk to the
instructor if you're concerned about your preparation.  Programming
assignments will use the programming language Lua and the Torch
framework. We do \textbf{not} expect any familiarity with Torch/Lua,
but think of it like NumPy or MatLab. We expect written assignments to
be submitted in LaTeX.

\paragraph{Textbook}

The course will not have a textbook. We will draw heavily from a set
of free course notes available online. For background in machine
learning, we recommend \textbf{Machine Learning: A Probabilistic
  Approach} which is available in the COOP, and is a generally
worthwhile reference. We will also read several research papers during
the term.  Finally, the course staff is always happy to recommend
additional readings or other sources of information if you would like
to explore a topic from the course in more depth.

\paragraph{Laptop Policy}

For the sake of cutting down on distraction and maintaining an
academic atmosphere, phones and laptops will in general not be
permitted during lecture. As we understand that some students prefer
laptops for note-taking, we ask that you contact the course staff at
the beginning of the semester if you require your laptop during
class. We will also ask that students using laptops sit in a designated
section so as not to distract other colleagues.

\paragraph{Support resources}

We will be using the Piazza for questions. Unless your question would
reveal confidential information or give away answers to homework
questions, please post there. We also encourage you to answer each
others questions.

\paragraph{Office hours} 
The staff office hours will be posted on the
website. You are welcome to come with specific questions about the
material, to discuss final project ideas, or just to chat about things
you find interesting and want to explore further.

\paragraph{Email} Staff emails are posted on the website. To avoid
duplication of questions and keep the email load manageable, please
use the forums if your question may be of interest to other students,
and only use email for personal questions.

\section{Provisional Schedule}

This weekly schedule is provisional. It may be adjusted based on the observed pace of the course:
\vspace {0.25cm}



 \begin{center}
\begin{tabular}{llll}
\toprule
Date &Topic &Lecture &Assignment \\
\midrule
Jan. 26 & \textit{Natural Language Processing}& Tasks & \\
Jan. 28 & Text Classification & Naive Bayes & HW 1 \\
Feb. 2  & & (Multinomial) Logistic Regression \\
Feb. 4  & & Optimization Methods \\
Feb. 9  & Neural Networks & Fundamentals & HW 2 \\
Feb. 11 & & Neural Text Classification & \\
Feb. 16 & & Convolutional Neural Networks &\\
Feb. 18 & Language Modeling & Multinomial Models & HW 3 \\
Feb. 23 & & Neural Language Modeling & \\
Feb. 26 & & Embeddings  & \\
Mar. 1 & Midterm  & \\
Mar. 3 & Application: Coreference \\
Mar. 8 & Recurrent Neural Networks & Fundamentals & HW 4 \\
Mar. 10 & & LSTMs and Language Modeling & \\
Mar. 22& & Approximate Search & \\
Mar. 24 & & LSTMs and Machine Translation & \\
Mar. 29 & Structured Prediction & Sequence Models &  \\
Mar. 31& & Filtering, Smoothing, Viterbi & HW 5 \\ 
Apr. 5 & & Conditional Random Fields 1 & \\
Apr. 7 & & Conditional Random Fields 2  \\
Apr. 12 & & Dependency Parsing \\
Apr. 14 & Topics & Lagrangian Relaxation  \\
Apr. 19 &  & Attention-Models \\
Apr. 21& & Guest Lecture \\
Apr. 26 & Final Project Presentations \\
\bottomrule
\end{tabular}
 \end{center}

\vspace {0.25cm}
\section{Course Requirements}

The course has several components:

\begin{itemize}
\item Six assignments; each will have a computational part and written part (40\%)
\item An in-class exam (10\%)
\item A final project  (40\%)
\item Paper discussion and scribing (10\%)
\end{itemize}


\noindent Final grades take into account each component. You must
achieve a passing grade in all components to pass this course. To
receive an A you must have high performance in all categories.

\paragraph{Assignments}

The 5 assignments (HW 1 - HW 5) will be published on the course
webpage. Most assignments have two components: computational and
written. Both parts should be done in pairs
and will require programming and experimentation. Computational assignments will ask you to develop
implementations of algorithms for multi-class classification, neural
network classification, language modeling, and generation with an
LSTM, and training a conditional random field. You be expected to
apply them to different real-world problems, and to analyze the
performance in a write-up.

\paragraph{Late days} Each student is allotted \textbf{five} late days
which may be applied to any of the assignments.  A late day extends the
due date by 24 hours. No more than two late days may be used on any
one assignment. In cases of medical or other emergencies which
interfere with your work, please contact the
instructor.

\paragraph{Grading} Assignments will be due at \textbf{11:59pm} on the day
scheduled. If you have used up your 5 late days, you will be penalized
25\% per day, up to two days max, with no credit after two days. We
will only give extensions for emergencies, and you will need a note
from either a doctor or your Resident Dean. Computational components
will be graded based on correctness, performance and documentation.
Written components will be graded based on correctness, depth of
analysis, and clarity.

% \paragraph{Readings}

% Each class meeting is preceded by a reading assignment. It is
% important to keep on top of the reading, which will be assumed during
% the lecture and discussion in class. You should set aside 2 hours to
% compete each reading. We do not expect you to fully understand
% everything before coming to class, but the goal is to prepare for
% class, familiarize yourself with new terminology and definitions, and
% to determine which part of the subject you want to hear more about.
% We encourage you to bring questions to class about material that is
% confusing.  Other students might share your confusion.

\paragraph{Participation, Discussion, and Scribing}

We will have several paper discussions sessions. You are expected to
read the paper before class and engage in discussion in the class
itself.  We will offer two sources of additional credit in the class
for participation. First for students who go out of their way to
provide support in the Piazza forums, and second for students who find
bugs or errata in the course lecture notes or homeworks. For the
second, please email the professor with the subject \texttt{CS287
  Errata} in the subject line. Finally each student will also be asked
to scribe one class in the semester. The scribe will be required to
provide a complete TeX'd version of the notes for that lecture.


\paragraph{In-class Exams}

In addition to homework assignments, there is one in-class exam
(closed book, no notes), covering the first half of the course
material. See the schedule for dates and topics covered. 

\paragraph{Final Project}

During the course students will design and carry out a final project,
working in pairs. The final project is of your choosing, but we expect
it to aspire to the level of a conference publication. We will provide
a list of potential topics and an opportunity to get feedback before
starting. The final presentation and paper (by the group)
are due at the end of reading period, and attendance at the final
presentation sessions is mandatory.

The project grade is based three aspects: 
\begin{enumerate}
\item project concepts and results 
\item presentation quality 
\item final paper quality 
\end{enumerate}

\paragraph{Collaboration Policy}

Each assignment will include a computational component and a written component.

The computational component of assignments 1-5 can done and submitted
in pairs. In pairs implies designing and writing the code together and
submitting a single assignment and receiving the same grade. Note that
we will treat pairs/non-pairs the same from a grading perspective.  We
expect you and your partner to design and implement the solutions
together. You may also consult with your classmates in other groups as
you work on the problem, but you should not talk in terms of
pseudocode or real code, and you should not share answers. In
addition, you must cite any books, articles, websites, lectures,
etc. that have helped you with your work. Similarly, you must list the
names of students from other groups with whom your group has
collaborated. If you are doing the computational assignment
individually, then the same rules apply for collaboration as for the
written assignments: talking is ok, sharing code is not.

The written component of all assignments must be done individually,
and each person must submit her/his own written assignment. You are
encouraged to consult with your classmates as you work on the problems
for the written assignments. However, you should not share answers.
After discussions with your peers, make sure that you can work through
the problem yourself and ensure that any answers you submit for
evaluation are the result of your own efforts. In addition, you must
cite any books, articles, websites, lectures, etc. that have helped
you with your work. Similarly, you must list the names of students
with who you have collaborated.  Note that understanding the concepts
in the written assignments is important both for the computational
components and the exams. Final projects must be done in pairs. You
are encouraged to discuss your project ideas with your peers.

For any questions not covered in this document, email the course staff
for clarification.



\end{document}