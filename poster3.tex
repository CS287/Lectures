\documentclass{article}

\usepackage{palatino}
\usepackage{listings}
\usepackage{color}
\usepackage{graphicx}
\title{CS287: Natural Language Processing}
\pagenumbering{gobble}
\date{}
\begin{document}


\begin{figure}[c!]
  \centering
\lstset{language=C++,
                basicstyle=\ttfamily,
                keywordstyle=\color{blue}\ttfamily,
                stringstyle=\color{red}\ttfamily,
                commentstyle=\color{magenta}\ttfamily,
                morecomment=[l][\color{magenta}]{\#}
}

\begin{lstlisting}
/*
 * Increment the size file of the new 
 * incorrect UI_FILTER group information
 * of the size generatively.
 */
static int indicate_policy(void)
{
  int error;
  if (fd == MARN_EPT) {
    /*
     * The kernel blank will coeld it to userspace.
     */
    if (ss->segment < mem_total)
      unblock_graph_and_set_blocked();
    else
      ret = 1;
    goto bail;
  }
  segaddr = in_SB(in.addr);
  selector = seg / 16;
  setup_works = true;
  for (i = 0; i < blocks; i++) {
    seq = buf[i++];
    bpf = bd->bd.next + i * search;
    if (fd) {
      current = blocked;
    }
  }
  rw->name = "Getjbbregs";
  bprm_self_clearl(&iv->version);
  regs->new = blocks[(BPF_STATS << info->historidac)] 
       | PFMR_CLOBATHINC_SECONDS << 12;
  return segtable;
}
\end{lstlisting}

\vspace{1cm}
\begin{center}
\begin{center}
  {\huge Learn to Code$^*$}
\end{center}
\vspace{1cm}

 {\LARGE CS287: Natural Language Processing}

 {\LARGE Tues/Thurs 2:30-4}
 \vspace{0.25cm}

 {\LARGE cs287.fas.harvard.edu}

\end{center}
\vspace{2cm}
{\footnotesize $^*$Hallucinated source-code generated by an LSTM language model trained on the complete Linux source code. From \textit{The Unreasonable Effectiveness of Recurrent Neural Networks}  (http://karpathy.github.io/2015/05/21/rnn-effectiveness/)}
\end{figure}


\end{document}



\end{document}
