\documentclass{beamer}
\usepackage{../common_slides}
\usepackage{tikz-qtree}

\title{Language Modeling \\ + \\ Feed-Forward Networks 3}
\date{}
\author{Alexander Rush}
\begin{document}

\begin{frame}
  \titlepage
\end{frame}

\begin{frame}{Quiz}
  Consider the hierarchical softmax from the last class.

  Say we have trained a depth-two balanced soft-max tree.
  Give an algorithm and time-complexity for:

  \begin{itemize}
  \item Computing the probability of a class?
  \item Computing the optimal class decision?
  \item Computing the full distribution over classes?
  \end{itemize}
\end{frame}


\begin{frame}{The Language Modeling }
  [The dog walked to the \alert{blank}]
\end{frame}

\begin{frame}{Language Modelling Interpretation}
  [from talk]
\end{frame}

\begin{frame}{Language Modeling}
  Crucially important for:

  \begin{itemize}
  \item Speech Recognition
  \item Machine Translation
  \item Many deep learning applications
    \begin{itemize}
    \item Captioning
    \item Dialogue
    \item Summarization
    \item $\dots$
    \end{itemize}
  \end{itemize}
\end{frame}

\begin{frame}{Speech Recognition }

\end{frame}

\begin{frame}{Machine Translation}

\end{frame}

\begin{frame}{Language Modeling Formally}
  \textbf{Goal:} Estimate a distribution of a sentence

  \begin{itemize}
  \item Factorization:
    \[ p(w_1, \ldots, w_n) = \prod_{t=1}^n p(w_t | w_1, \ldots, w_{t-1}) \]
  \end{itemize}

  Estimate this probability distribution:
  \[ p(w_t | w_1, \ldots, w_{t-1};\theta)\]
\end{frame}

\begin{frame}{Machine Learning Setup}
  \[ (\boldx_1, \boldy_1), \ldots, (\boldx_n, \boldy_n) \]
  \begin{itemize}
  \item $\boldy_i$; the one-hot next word
  \item $\boldx_i$; representation of the prefix $(w_1, \ldots, w_{t-1})$
  \end{itemize}
\end{frame}

\begin{frame}{Metric}
  So far, we have used \textit{accuracy} as a metric.

  Language modeling uses test average negative log-likelihood (cross-entropy)
  \begin{itemize}
    \item For Test data $\bar{w}_1, \ldots, \bar{w}_n$
    \item $\frac{1}{n}\sum_{i=1}^n \log y_c$
  \end{itemize}


  Actually report \textit{perplexity}
  \[ \exp(\frac{1}{n}\sum_{i=1}^n \log y_c) \]
\end{frame}

\begin{frame}{Perplexity: Intuition}
  \begin{itemize}
  \item Effective uniform distribution size.
  \item If words were uniform: $|\mcV| = 10000$
  \end{itemize}
\end{frame}


\begin{frame}{Bigram Models}
  \begin{itemize}
  \item Feature set $\mcF$: previous words
  \item Input vector $\boldx$ is sparse
  \item Count matrix $\boldF \in \reals^{\mcV \times \mcV} $
  \end{itemize}

  \[ F_{v, w}  =  \indicator(w_{i-1} = w, w_i = v)  \]

  \[p(\boldy | \boldx;\theta) = \frac{F_{v, w}}{\sum_{v'} F_{v', w}} \]
\end{frame}


\begin{frame}{Trigram Models}
  \begin{itemize}
  \item Feature set $\mcF$: previous two words (conjunction)
  \item Input vector $\boldx$ is sparse
  \item Count matrix $\boldF \in \reals^{\mcV \times \mcV, \mcV} $
  \end{itemize}

  \[ F_{u, v, w}  = \sum_{} \indicator(w_{i-2} = w, w_{i-1} = w, w_i = v)  \]

  \[p(\boldy | \boldx;\theta) = \frac{F_{u, w, v}}{\sum_{w'', w'} F_{w'', w',v}} \]
\end{frame}


\begin{frame}{NGram Models}
  It is common to go up to 5-grams

  \[ F_{c, w}  =  \sum_{} \indicator(w_{i-n+1} \ldots w_{i-1} = c, w_{i} = w)  \]

  \[ N_{c, w}  =  \indicator(F_{c,w} > 0)  \]


  \begin{itemize}
  \item Google n-grams goes up ... [link]
  \end{itemize}
\end{frame}

\begin{frame}{Notation}
  \begin{itemize}
  \item $N_{\cdot, w} = \sum_c N_{c,w} $
  \item $N_{c, \cdot} = \sum_w N_{c,w} $
  \item $c_{2:}$; context without first word
  \item $p_{ML}(w | c) = N_{w, c} / N_{c, \cdot}$
  \end{itemize}
\end{frame}

\begin{frame}{Additive Smoothing}
  Laplace Smoothing
  \[\alpha \]
\end{frame}

\begin{frame}{Interpolation (Jelinek-Mercer Smoothing)}
  \[ p_{interp}(w |  c) =  \lambda p_{ML}(w |  c) + (1 - \lambda) p_{interp}(w | c_{2:}) \]

  Ensure that $\lambda$ form convex combination
  \[\lambda \geq 0\]

\end{frame}


\begin{frame}{Interpolation (Witten-Bell)}

  Defind $\lambda$ as a function of $w_{i-n+1} \ldots w_i$

  \[(1 - \lambda) =  \frac{\displaystyle N_{c, \cdot}{\displaystyle N_{c, \cdot} + F_{c, w}}\]


  \[ p_{wb}(w | c) = \frac{\displaystyle F_{c, w} + N_{c, \cdot} \times p_{wb}(w| c_{2:})} {\displaystyle N_{c, \cdot}) + F_{c, \cdot}} \]

    Interpolation counts are a new events estimated as proportional to seeing a new word.
\end{frame}

\begin{frame}{Absolute Discounting}
  Similar form to interpolation

  \[ p_{discount} =\frac{\max{0, F_{c,w} - D}}{F_{c, \cdot} +  (1 - \lambda) p_{discount}(w |  c_{2:} )  \]

  Where

  \[(1-\lambda) = \frac{D}{F_{c, \cdot} N_{c,\cdot} \]
\end{frame}

\begin{frame}{Review: Marginalization}

\end{frame}

\begin{frame}{Kneser-Ney Smoothing}
  Idea: match interpolated marginals
  \[ \sum_{c_1} p_{KN}(c_1, c_{2:}, w) =  \sum_{c_1} p_{KN}(w |  c) p_{ML}(c)  = p_{ML}(c_{2:}, w)   \]

  \[ \sum_{c_1} p_{KN}(w |  c) F_{c,\cdots}  = F_{c_{2:}w, \dots}  \]

  \begin{eqnarray*}
    F_{c_{2:}w, \dot} &=& \sum_c F_{c,\cdots} [p_{KN}(w,c)]  \\
    &=& \sum_c F_{c,\cdots} [\frac{\max{0, F_{c,w} - D}}{F_{c, \cdot} +  \frac{D}{F_{c, \cdot} N_{c,\cdot} p_{discount}(w |  c_{2:} )]  \\

    &=& \sum_{c_1:N_{c_1 c_2, w} > 0}  F_{c,\cdots} \frac{F_{c,w} - D}{F_{c, \cdots}} + \sum_{c_1} \frac{D}{F_{c, \cdots}} N_{\cdot c_{2:}, \cdots} p_{KN}(w| c_{2:})
    &=& F_{c_{2:}w, \cdot} - N_{\cdots c_{2:}, w}D + D p_{KN}(w| c_{2:}) N_{\cdots c_{2:}, \cdots}
  \end{eqnarray*}

\end{frame}

\begin{frame}{Kneser-Ney Smoothing}
  Final equation

  \begin{eqnarray*}
    p_{KN}(w| c_{2:}) & = & N_{\cdots c_{2:}, w} /  N_{\cdots c_{2:}, \cdots}
  \end{eqnarray*}

\end{frame}

\section{}

\begin{frame}{Sparse: Efficiency Issues}
  Very quickly the count matrix

  Reverse trie data structure
  \begin{itemize}
  \item SRILM
  \end{itemize}
  \Tree [ .ROOT .runs()  [  .dog()  .the() ]   ] ;
\end{frame}

\begin{frame}{Efficiency: Hash }
\end{frame}

\begin{frame}{Efficiency: Reverse Tries }
  Of course each is hashed

  Reverse trie data structure
  \begin{itemize}
  \item SRILM, KenLM
  \end{itemize}
  \Tree [ .ROOT .runs()  [  .dog()  .the() ]   ] ;
\end{frame}

\begin{frame}{Efficiency: Quantization }


\end{frame}


\end{document}